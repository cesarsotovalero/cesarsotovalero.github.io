\documentclass[a4paper]{article}

\usepackage[utf8]{inputenc}
\usepackage[T1]{fontenc}
\usepackage{a4wide}
\usepackage{graphicx}
\usepackage{framed}
\usepackage{url}
\usepackage{natbib}
\usepackage{amssymb}
\usepackage{subfigure}
\usepackage{morefloats}
\usepackage{multirow}
\usepackage{lineno}
\usepackage{amsmath}
\usepackage{booktabs}
\usepackage{xspace}
\usepackage{array}
\usepackage{stfloats}
\usepackage{tabularx}
\usepackage{xspace}
\usepackage{graphicx}
\usepackage{epstopdf}
\usepackage{wrapfig}
\usepackage{pbox}
\usepackage{tcolorbox,tikz}
\usepackage{soul}
\usepackage{pgfplots}
\usepackage{pgfplotstable}
\usepackage{xcolor}
\usepackage{listings}
\usepackage{ragged2e}
\usepackage{amsmath}
\usepackage{makecell}
\usepackage{filecontents}
\usepackage[scale=0.95877]{tgpagella} % use "palatino" as main font, ...
\renewcommand{\ttdefault}{lmtt}       % and use lmtt for teletype family

\pgfplotsset{compat=1.14}
\let\subfigure=\undefined

\newenvironment{citebox}
{\begin{framed}\begin{sffamily}}
{\end{sffamily}\end{framed}\vspace{2mm}}

\usepackage[
    pdfpagemode={UseOutlines},
    bookmarks,
    hypertexnames=false,
    colorlinks,
    linkcolor={blue},
    citecolor={blue},
    urlcolor={blue},
    pdfstartview={FitH}
] {hyperref}

\usepackage[
    protrusion=true,
    activate={true,nocompatibility},
    final,
    tracking=true,
    kerning=true,
    spacing=false,
    factor=1100]{microtype}
\SetTracking{encoding={*}, shape=sc}{40}

% -----------------------------------------------------------------------------
% Create custom user defined command for `TODO` notes
\newcommand*\badge[1]{
  \colorbox{red}{\color{white}#1}
}
\newcommand{\todo}[1]{
  \noindent\textbf{\badge{TODO}} {\color{red}#1}
  \GenericWarning{}{LaTeX Warning: TODO: #1}
}

\urlstyle{sf}

\pgfplotsset{
    /pgf/declare function={
        Floor(\x) = round(\x-0.49);
    },
    show sum on top/.style={
        /pgfplots/scatter/@post marker code/.append code={%
            \path let \p1=($(normalized axis cs:%
                        \pgfkeysvalueof{/data point/x},%
                        \pgfkeysvalueof{/data point/y})%
                        -(normalized axis cs:\pgfkeysvalueof{/data point/x},0)$)
            in node[
                at={(normalized axis cs:%
                        \pgfkeysvalueof{/data point/x},%
                        \pgfkeysvalueof{/data point/y})%
                },
                anchor={-90*sign(\y1)},yshift={sign(\y1)*2pt}
            ]
            {\pgfmathprintnumber{\pgfkeysvalueof{/data point/y}}};
        },
    }
}

\newcommand{\proglang}{\textbf}
\newcommand{\pkg}{\textbf}
\newcommand{\code}{\texttt}
\newcommand{\modif}[1]{{\color{blue}#1}}
\newcommand{\Fbox}[1]{\fbox{\strut#1}}

% -----------------------------------------------------------------------------

\begin{document}
\begin{center}
{\Large\bf \textcolor{red}{SUBMISSION-ID}: Major Revision}\\\vspace{4mm}
\end{center}

\noindent Dear Editorial Office of \textcolor{red}{Journal Name},\\

We are happy to submit a minor revision of our paper entitled \textcolor{red}{\textbf{“Paper's Title\textbf{”}}}.
We thank the reviewers for their recommendations.
We have addressed all their comments, which has improved the quality of our manuscript.\\

\noindent  We performed \textcolor{red}{N} minor revisions:

\begin{itemize}
    \item \textcolor{red}{Summarize major change here}
    \item \textcolor{red}{Summarize major change here}
    \item \textcolor{red}{Summarize major change here}
\end{itemize}

In the following pages, we give detailed answers to each of the reviewers' comments. The original text from the reviewers is included in \fbox{boxes}, our answers follow the boxes. All changes are highlighted in \modif{blue} in the revised version of the manuscript (except typos).\\

\noindent In case of requiring any further information, please do not hesitate to contact us.\\

\bigbreak

\noindent  Sincerely yours,\\

\noindent \textcolor{red}{Full name of the first author}\\

\noindent On behalf of \textcolor{red}{full names of co-authors}

\newpage

% -----------------------------------------------------------------------------

\subsection*{Reviewer \#1}

\begin{citebox}
\textbf{Comment \#1.1:} “\textcolor{red}{Comment of Reviewer \#1}”
\end{citebox}

\textcolor{red}{Answer the comment here.}

\subsection*{Reviewer \#2}

\begin{citebox}
\textbf{Comment \#2.1:} “\textcolor{red}{Comment of Reviewer \#2}”
\end{citebox}

\textcolor{red}{Answer the comment here.}

\subsection*{Reviewer \#3}

\begin{citebox}
\textbf{Comment \#3.1:} “\textcolor{red}{Comment of Reviewer \#3}”
\end{citebox}

\textcolor{red}{Answer the comment here.}

\section*{Bonus}

You should use quoted text to highlight changes in the paper:

\begin{quote}
\textit{
“The distance from Foo to Bar is $ 0.42 $.”
}
\end{quote}


\noindent Make sure to add proper citations, e.g.~\cite{Page1999}.

\noindent You can also use TODO notes to track the progress, e.g., \todo{A note!}

% -----------------------------------------------------------------------------

\begin{filecontents}{my_references.bib}
@techreport{Page1999,
  title={The PageRank Citation Ranking: Bringing Order to The Web.},
  author={Page, Lawrence and Brin, Sergey and Motwani, Rajeev and Winograd, Terry},
  year={1999},
  institution={Stanford InfoLab}
}
\end{filecontents}

\bibliographystyle{plain}
\bibliography{my_references}

\end{document}